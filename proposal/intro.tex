\section{Introduction}
\label{sec:intro}

In an era where digital transactions and interactions have become ubiquitous, the significance of robust authentication mechanisms cannot be overstated. The security of online systems heavily relies on the ability to accurately verify the identity of users. Traditional authentication methods ~\cite{idrus2013reviewauth}, which include passwords, security tokens, and biometric systems, aim to balance convenience with security. However, they often fall short when it comes to protecting users' privacy. Personal data breaches, identity theft, and unauthorized surveillance are prevalent issues that stem from the overexposure of personally identifiable information (PII) during the authentication process ~\cite{wang2021authattacks}.

The central dilemma is thus: How can we verify an individual's identity and grant access to services without compromising their privacy by revealing or storing sensitive personal information? This question has sparked considerable interest in the field of cryptography and, more specifically, in the concept of zero-knowledge proofs (ZKPs) ~\cite{goldwasser1989zkp}. ZKPs are a breakthrough in cryptographic techniques that allow one party (the prover) to prove to another party (the verifier) that a given statement is true, without revealing any information beyond the validity of the statement itself. This can be likened to proving that a key fits a lock without revealing the shape of the key.

The motivation behind this project is to harness the power of zero-knowledge proofs to create a privacy-preserving authentication system. Such a system would enable users to prove their identity or credentials without exposing any PII, thus maintaining their privacy and security. The potential applications of this technology are vast and include sectors where privacy is paramount, such as online banking, secure communications, e-voting systems, and healthcare.

The challenge we aim to address is: firstly, to design and implement an authentication system that is as intuitive and user-friendly as traditional methods, and secondly, to ensure this system upholds the rigorous security standards required for widespread adoption without incurring prohibitive computational costs.

To achieve this goal, we will need to navigate a landscape where the theoretical possibilities of zero-knowledge proofs intersect with practical limitations. We will explore the latest advancements in zero-knowledge protocols, such as zk-SNARKs (zero-knowledge Succinct Non-interactive Arguments of Knowledge) ~\cite{petkus2019zksnark,chen2023reviewzksnark} and zk-STARKs (zero-knowledge Scalable Transparent Arguments of Knowledge) ~\cite{berentsen2022walkthroughzkstark,cryptoeprint2018zkstark} and investigate how these can be optimized for real-world authentication scenarios. This exploration will include addressing potential scalability issues and reducing the computational overhead to facilitate a seamless user experience.

Our project, therefore, stands at the intersection of cryptography, software engineering, and usability design. It is poised to make a significant contribution to the field of cybersecurity by providing a solution that not only protects users' digital identities but also upholds the fundamental right to privacy in the digital realm. With the ever-increasing value placed on personal data, the importance of such a system cannot be underestimated. The successful implementation of a zero-knowledge proof-based authentication system will represent a paradigm shift in how we approach security and privacy in an interconnected world.
