\section{Evaluation Plan}
\label{sec:eval-plan}

To rigorously assess the proposed system, we will employ a multi-dimensional evaluation plan encompassing qualitative and quantitative metrics.

\subsection{Evaluation Metrics}
\label{subsec:eval-metrics}

\begin{itemize}
    \item \textbf{Authentication Success Rate (ASR)}: The ratio of successful authentications to the total authentication attempts.
    \item \textbf{Authentication Time}: Duration from the initiation to completion of the authentication process.
    \item \textbf{System Efficiency}: Computational and memory resources utilized during authentication.
    \item \textbf{Security Analysis}: System robustness against common attack vectors.
    \item \textbf{User Satisfaction}: User experience assessed through surveys and interviews.
    \item \textbf{Scalability}: The system's ability to handle increased load effectively.
\end{itemize}

\subsection{Data Collection}
\label{subsec:data-collection}

Data will be collected through system logs, user feedback, and security incident reports to provide a comprehensive evaluation of the system's performance.

\subsection{Evaluation Methodology}
\label{subsec:eval-method}

A multi-stage evaluation approach, including laboratory testing, field testing, A/B testing, security auditing, and scalability testing, will be utilized to ensure a thorough assessment.
