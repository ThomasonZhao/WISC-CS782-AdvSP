\section{Introduction}
\label{sec:intro}

The widespread reliance on password-based authentication has long been a contentious topic in the cybersecurity landscape. Traditional password-based authentication systems are susceptible to various attacks, such as password guessing, dictionary attacks, and credential stuffing, which can compromise user accounts and expose sensitive information. To address these vulnerabilities, researchers have developed more robust authentication methods, with password-authenticated key exchange (PAKE) protocols emerging as a promising solution.

PAKE protocols are designed to establish a secure session key between a client and a server without revealing the user's password, even in the face of an adversary with extensive knowledge of the system. These protocols leverage cryptographic techniques to ensure that after a login attempt, whether valid or invalid, the client and server only learn whether the password matched the expected value, without leaking any additional information. This property is particularly valuable in threat models where adversaries may have access to password leaks, password distributions, and substantial computational power.

The primary goal of this project was to evaluate the usability of different PAKE protocols, assessing their performance, scalability, and user experience. By conducting a comprehensive analysis, we aimed to provide insights into the practical implications of adopting PAKE protocols in real-world authentication systems, addressing the tradeoffs between security, efficiency, and user-friendliness.

To achieve this objective, we implemented three PAKE protocols - Password Over TLS, Secure Remote Password (SRP), and OPAQUE - within a web-based application framework utilizing React.js for the front-end, Node.js for the back-end, and MongoDB for the database. We then performed a series of experiments to measure the protocols' performance and scalability, focusing on metrics such as CPU usage, memory consumption, and storage requirements. Additionally, we conducted a user experience study to gather feedback on the usability, trustworthiness, and perceived safety of the PAKE protocols from a diverse group of participants.

The findings from this project contribute to the understanding of the practical considerations surrounding the deployment of PAKE protocols in real-world settings. By evaluating the tradeoffs between the security benefits and the system-level impact of these protocols, we hope to inform decision-makers and system designers on the feasibility and trade-offs of incorporating PAKE-based authentication into their applications, ultimately enhancing the overall security posture while maintaining a positive user experience.